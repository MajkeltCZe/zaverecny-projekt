% ŠABLONA PRO PSANÍ ZÁVĚREČNÉ STUDIJNÍ PRÁCE
%%%%%%%%%%%%%%%%%%%%%%%%%%%%%%%%%%%%%%%%%%%%
% Autor: Jakub Dokulil (kubadokulil99@gmail.com)
% Tato šablona byla vytvořena tak, aby pomocí ní mohli v systému LaTeX soutěžící sázet své práce a zároveň odpovídala požadavkům na formátování vyplývajícím z wordové šablony umístěné na webu soc.cz.
%
\documentclass[12pt, a4paper,
%oneside,      %% -- odkomentujte, pokud chcete svou práci mít pouze jednostrannou, mezera pro hřbet pak automaticky bude pouze na levé straně
twoside,        %% -- pro oboustranné práce, mezera pro hřbet následně střídá strany.
openright
]{report}

%% Nutné balíčky a nastavení
%%%%%%%%%%%%%%%%%%%%%%%%%%%%

%% Proměnné
\newcommand\obor{INFORMAČNÍ TECHNOLOGIE} %% -- napiš číslo a název tvého oboru
\newcommand\kodOboru{18-20-M/01} %% -- napiš číslo a název tvého oboru
\newcommand\zamereni{se zaměřením na počítačové sítě a programování} %% -- napiš číslo a název tvého oboru
\newcommand\skola{Střední škola průmyslová a umělecká, Opava} %% vyplň název školy
\newcommand\trida{IT4} %% vyplň jméno svého konzultanta
\newcommand\jmenoAutora{Michaela Říčná}  %% vyplň své jméno
\newcommand\skolniRok{2023/24} %% vyplň rok
\newcommand\datumOdevzdani{1. 1. 2024} %% vyplň rok
\newcommand\nazevPrace{Mobilní aplikace s využitím rozšířené reality} %% vyplň název své práce

\title{\nazevPrace} %% -- Název tvé práce
\author{\jmenoAutora} %% -- tvé jméno
\date{\datumOdevzdani} %% -- rok

\usepackage[top=2.5cm, bottom=2.5cm, left=3.5cm, right=1.5cm]{geometry} %% nastaví okraje, left -- vnitřní okraj, right -- vnější okraj

\usepackage[czech]{babel} %% balík babel pro sazbu v češtině
\usepackage[utf8]{inputenc} %% balíky pro kódování textu
\usepackage[T1]{fontenc}
\usepackage{cmap} %% balíček zajišťující, že vytvořené PDF bude prohledávatelné a kopírovatelné

\usepackage{graphicx} %% balík pro vkládání obrázků

\usepackage{subcaption} %% balíček pro vkládání podobrázků

\usepackage{hyperref} %% balíček, který v PDF vytváří odkazy

\linespread{1.25} %% řádkování
\setlength{\parskip}{0.5em} %% odsazení mezi odstavci


\usepackage[pagestyles]{titlesec} %% balíček pro úpravu stylu kapitol a sekcí
\titleformat{\chapter}[block]{\scshape\bfseries\LARGE}{\thechapter}{10pt}{\vspace{0pt}}[\vspace{-22pt}]
\titleformat{\section}[block]{\scshape\bfseries\Large}{\thesection}{10pt}{\vspace{0pt}}
\titleformat{\subsection}[block]{\bfseries\large}{\thesubsection}{10pt}{\vspace{0pt}}


\usepackage{tocloft} % Balíček umožní přizpůsobit vzhled tabulky obsahu
\setlength{\cftbeforechapskip}{0pt}  % Menší rozestup pro kapitoly
\setlength{\cftbeforesecskip}{0pt}   % Menší rozestup pro sekce

\setcounter{secnumdepth}{2}
\setcounter{tocdepth}{1}
\usepackage{fancyhdr}
\pagestyle{fancy}
\renewcommand{\headrulewidth}{0.025pt}

\usepackage{booktabs}

\usepackage{url}

%% Balíčky co se můžou hodit :) 
%%%%%%%%%%%%%%%%%%%%%%%%%%%%%%%

\usepackage{pdfpages} %% Balíček umožňující vkládat stránky z PDF souborů, 

\usepackage{upgreek} %% Balíček pro sazbu stojatých řeckých písmen, třeba u jednotky mikrometr. Například stojaté mí: \upmu, stojaté pí: \uppi

\usepackage{amsmath}    %% Balíčky amsmath a amsfonts 
\usepackage{amsfonts}   %% pro sazbu matematických symbolů
\usepackage{esint}     %% pro sazbu různých integrálů (např \oiint)
\usepackage{mathrsfs}
\usepackage{helvet} % Helvet font
\usepackage{mathptmx} % Times New Roman
\usepackage{Oswald} % Oswald font


%% makra pro sazbu matematiky
\newcommand{\dif}{\mathrm{d}} %% makro pro sazbu diferenciálu, místo toho
%% abych musel psát '\mathrm{d}' mi stačí napsat '\dif' což je mnohem 
%% kratší a mohu si tak usnadnit práci

\usepackage{listings}
\usepackage{xcolor}

\renewcommand{\lstlistingname}{Kód}% Listing -> Algorithm
\renewcommand{\lstlistlistingname}{Seznam programových kódů}% List of Listings -> List of Algorithms

%% Definice 
%copied code start
\definecolor{bluekeywords}{rgb}{0,0,1}
\definecolor{greencomments}{rgb}{0,0.5,0}
\definecolor{redstrings}{rgb}{0.64,0.08,0.08}
\definecolor{xmlcomments}{rgb}{0.5,0.5,0.5}
\definecolor{types}{rgb}{0.17,0.57,0.68}


\lstset{language=[Sharp]C,
	captionpos=b,
	%numbers=left, %Nummerierung
	%numberstyle=\tiny, % kleine Zeilennummern
	frame=lines, % Oberhalb und unterhalb des Listings ist eine Linie
	showspaces=false,
	showtabs=false,
	breaklines=true,
	showstringspaces=false,
	breakatwhitespace=true,
	escapeinside={(*@}{@*)},
	commentstyle=\color{greencomments},
	morekeywords={partial, var, value, get, set},
	keywordstyle=\color{bluekeywords},
	stringstyle=\color{redstrings},
	basicstyle=\ttfamily\small,
}
% copied code end

% Nastavení barev
% Requires package: color.
\definecolor{mediumgray}{rgb}{0.3, 0.4, 0.4}
\definecolor{mediumblue}{rgb}{0.0, 0.0, 0.8}
\definecolor{forestgreen}{rgb}{0.13, 0.55, 0.13}
\definecolor{darkviolet}{rgb}{0.58, 0.0, 0.83}
\definecolor{royalblue}{rgb}{0.25, 0.41, 0.88}
\definecolor{crimson}{rgb}{0.86, 0.8, 0.24}




%% Bordel pro práci - můžeš smáznout :) 
%%%%%%%%%%%%%%%%%%%

\usepackage{lipsum} %% balíček který píše lipsum (nesmyslný text, který se používá pro kontrolu typografie)

%% Začátek dokumentu
%%%%%%%%%%%%%%%%%%%%
\begin{document}
	
	\pagestyle{empty}
	\pagenumbering{Roman}
	
	\cleardoublepage

%% Titulní stránka s informacemi
%%%%%%%%%%%%%%%%%%%%%%%%%%%%%%%%%%%%%%%%
	
	{\fontfamily{phv}\selectfont
		%% Logo školy
		\begin{figure}[h]
			\centering
			\includegraphics[width=0.6\linewidth]{image/logo-skoly.png} 
		\end{figure}
		
		
		%% Hlavička práce a její název (viz proměnná \nazev prace)
		%% \sffamily %%% bezpatkové písmo - sans serif
		{\bfseries %%% písmo na stránce je tučně
			\begin{center}
				\vspace{0.025 \textheight}
				\LARGE{ZÁVĚREČNÁ STUDIJNÍ PRÁCE}\\
				\large{dokumentace}\\
				\vspace{0.075 \textheight}
				\LARGE {\nazevPrace}\\
			\end{center}  
		}%%%
		
		\begin{figure}[h]
			\centering
			\includegraphics[width=0.5\linewidth]{image/logo.png} 
		\end{figure}
		
		\vspace{0.02 \textheight}
		\begin{table}[h!]
			\begin{tabular}{ll}
				\textbf{Autor:} & \jmenoAutora\\ 
				\textbf{Obor:} & \kodOboru { } \obor\\
				\textbf{} & \zamereni\\
				\textbf{Třída:} & \trida\\
				\textbf{Školní rok:} & \skolniRok\\
			\end{tabular}
			
		\end{table}		
	}
	
\cleardoublepage %% Zalomení dvojstránky
	
%% Stránka obsahující poděkování 
%%%%%%%%%%%%%%%%%%%%%%%%%%%%%%%%%%%%%%%%%%%%%%%%%%%%%%%%

%% Poděkování - nepovinné
%%%%%%%%%%%%%%%%%%%%%%%%%%%%
	
	\noindent{\large{\bfseries{Poděkování}\\}}
	\noindent 
	
	\vspace*{0.7\textheight} %% Vertikální mezeru je možné upravit

%% Prohlášení - povinné
%%%%%%%%%%%%%%%%%%%%%%%%%%%%
	\noindent{\large{\bfseries{Prohlášení}\\}}  %% uprav si koncovky podle toho na jaký rod se cítíš, vypadá to pak lépe :) 
	\noindent{Prohlašuji, že jsem závěrečnou práci vypracovala samostatně a uvedla veškeré použité 
		informační zdroje.\\}
	\noindent{Souhlasím, aby tato studijní práce byla použita k výukovým a prezentačním účelům na Střední průmyslové a umělecké škole v Opavě, Praskova 399/8.}
	\vfill
	\noindent{V Opavě \datumOdevzdani\\}
	\noindent
	\begin{minipage}{\linewidth}
		\hspace{9.5cm} 
		\begin{tabular}{@{}p{6cm}@{}}
			\dotfill \\
			Podpis autora
		\end{tabular}
	\end{minipage}
	
	\cleardoublepage %% Zalomení dvojstránky

%% Stránka obsahující abstrakt (anotaci)
%%%%%%%%%%%%%%%%%%%%%%%%%%%%%%%%%%%%%%%%%%%%%%%%%%%%%%%%	

%% Abstrakt v češtině
%%%%%%%%%%%%%%%%%%%%%%%%%%%%
	\noindent{\Large{\bfseries{Abstrakt}\\}}
	\noindent Sem napíšeš svůj abstrakt.\\

	Délka cca 100 – 250 slov
	
	\vspace{18pt}
	
	\noindent{\large{\bfseries{Klíčová slova}}}
	
	\noindent Rozšířená realita, snímání obrázků, Unity, 3D objekty, mobilní aplikace
	
	\vspace{18pt}

%% Abstrakt v angličtině
%%%%%%%%%%%%%%%%%%%%%%%%%%%%	
	\noindent{\Large{\bfseries{Abstract}}}
	
	\noindent  %% přepiš!!
	
	\vspace{18pt}
	
	\noindent{\large{\bfseries{Keywords}}}
	
	\noindent Augmented Reality, image tracking, Unity, 3D objects, mobile application
	
	\clearpage %% Zalomení stránky

%% Stránka s generovaným obsahem
%%%%%%%%%%%%%%%%%%%%%%%%%%%%%%%%%%%%%%%	
	
	\tableofcontents %% Vygeneruje tabulku s obsahem

	\pagenumbering{arabic} %% Nastavení způsobu číslování stránek (alternativy roman | Roman)
	\setcounter{page}{1} %% Nastavení počitadla stránek

%% Stránka s úvodem - povinná část
%%%%%%%%%%%%%%%%%%%%%%%%%%%%%%%%%%%%%%%		
	\chapter*{Úvod}
	\addcontentsline{toc}{chapter}{Úvod}




 
%* náhled do řešené problematiky, zdůvodnění volby problematiky, 
%* předem definované cíle práce, 
%* motivaci pro další čtení textu včetně stručného uvedení obsahu následujících kapitol 


\chapter{Rozšířená realita}

\section{Co je to rozšířená realita}
\label{sec:co_je_AR}
Rozšířená realita (zkratka AR = augmented reality). Princip fungování rozšířené reality je v podstatě velmi jednoduchý – do obrazu reálného světa, který snímáme mobilním telefonem, tabletem či dalším zařízením, vkládáme navíc virtuální prvky – např. 3D model, video, textový či grafický popis, animace apod.


\section{Využití rozšířené reality}
\label{sec:vyuziti_AR}


\begin{itemize}
	\item Interaktivní vzdělávání,
	\item AR pro obchod,
	\item Návrh a vizualizace produktu,
	\item Vzdálená podpora v terénu,
	\item Hry a zábava.
\end{itemize}


\section{Typy rozšířené reality}
\label{sec:typy_AR}


\subsection{na základě značek}
Tento typ rozšířené reality využívá značky nebo také markery, když se určitý marker naskenuje, objeví se u něj digitální objekt. Markery mohou být jak QR kódy nebo obrázky.
Je důležité, aby marker, byl unikátní a dobře rozpoznatelný pro naskenování. 

\paragraph{Výhody}
\begin{itemize}
	\item Jednoduché pro začínající uživatele rozšířené reality.
	\item Snímání obrázků je stabilní. 
	\item Minimální výrobní náklady.
\end{itemize}

\paragraph{Nevýhody}
\begin{itemize}
	\item Funguje pouze v dostatečné blízkosti od kamery.
	\item Odraz světla na markeru, může způsobit problémy se snímáním.
	\item Aplikace potřebuje předem vytvořenou knihovnu referenčních markeru, pro spuštění.
\end{itemize}

\subsection{bez značek}
Nevyužívá značky k zobrazení obsahu rozšířené reality.

\subsubsection{na základě projekce}
Jedná se o jednoduchou formu rozšířené reality. Interakce probíhá fyzickým dotykem s projekčním povrchem. Mezi nejčastěji projekční povrchy se řadí zdi nebo podlahy.

\subsubsection{na základě lokace}
Díky dostupnosti chytrých telefonů využívat GPS jsou informace a virtuální objekty zobrazovány, když zařízení uživatele odpovídá konkrétnímu místu. 

	

\chapter{Využité technologie}
	
\section{Unity}
\label{sec:unity}
Multiplatformní herní engine, který nabízí spoustu balíčků, templatů a nástrojů pro vytváření nejrůznějších aplikací. Celá má aplikace byla za pomoci Unity vytvořena. Scripty byly napsány v C\#.

\subsection{AR Foundation}
Unity nabízí možnost využití templatu. AR Foundation je specifický template určený pro vytváření rozšířené reality. Obsahuje několik balíčků, které ulehčují práci s AR. 


\subsection{Balíčky}
V Unity je možnost stažení nejrůznějších balíčku, které mohou obsahovat scripty nebo objekty. Některé tyto balíčky se mohou stáhnout hned při vytvoření projektu, při použití templatu, nebo stáhnout později v Package Managaru.

\subsubsection{Quick Outline}
Tento balíček se využívá k vytváření obrysů pro jakékoliv objekty. Původně byl vyvinut pouze pro virtuální realitu, ale funguje i mimo ni. 


\subsubsection{XR Interaction Toolkit}
Balíček umožňuje lepší interakci s objekty. Součástí je Interactive Manager, ve kterém se dají nastavit různé možnosti interakce. 



\section{Modelovací programy}
\label{sec:modelovaci_programy}

Všechny modely jsem vytvářela sama. Buď za použití Blenderu nebo 3Ds Maxe. 




\chapter{Způsoby řešení a použité postupy}	
	
\section{Testovací aplikace}
\label{sec:testovaci_aplikace}

\section{Nastavení projektu}
\label{sec:nastaveni_projektu}
\section {Sledování obrázku}
\label{sec:sledovani_obrazku}







\section {3D objekty}
\label{sec:3d_objekty}
\section {Canvas}
\label{sec:canvas}
\section{Animace}
\label{sec:animace}

\section {Výsledky řešení}
\label{sec:vysledky_reseni}


	

	

	
\chapter*{Závěr}
\addcontentsline{toc}{chapter}{Závěr}




%% Seznam použitých informačních zdrojů
\renewcommand\bibname{Seznam použitých informačních zdrojů}
\begin{thebibliography}{99}
\addcontentsline{toc}{chapter}{Seznam použitých informačních zdrojů}
\bibitem{ARImageTracking} \textit{AR Foundation Improved Image Tracking - Multiple Objects/Images - Unity Augmented Reality/AR} [online]. YouTube, 5.4.2020 [cit. 2023-12-18]. Dostupné z: \url{https://www.youtube.com/watch?v=I9j3MD7gS5Y}

\bibitem{animations} \textit{Create UI ANIMATIONS without CODING! - Unity UI tutorial} [online]. YouTube, 24.3.2021 [cit. 2023-12-18]. Dostupné z: \url{https://www.youtube.com/watch?v=br9YzpiBeIw}	

\bibitem{ARvyuziti}\textit{Hospodářské noviny: Pět oblastí, kde můžete nejlépe využít rozšířenou realitu} [online]. [cit. 2023-12-17]. Dostupné z: \url{https://hn.cz/c1-66644350-pet-oblasti-kde-muzete-nejlepe-vyuzit-rozsirenou-realitu}

\bibitem{variables} \textit{How to get a variable from another script in Unity (the right way)} [online]. YouTube, 6.7.2022 [cit. 2023-12-18]. Dostupné z: \url{https://www.youtube.com/watch?v=2pCkInvkwZ0}

\bibitem{ARDrawing}\textit{Medium: 
How to Create AR Draw/Doodling in Unity3D} [online]. [cit. 2023-12-17]. Dostupné z: \url{https://medium.com/antaeus-ar/how-to-create-ar-draw-doodling-in-unity3d-ar-foundation-233b0e0f921e}

\bibitem{Onirix}\textit{Onirix: What Are The Different Types of Augmented Reality?} [online]. [cit. 2023-12-17]. Dostupné z: \url{https://www.onirix.com/learn-about-ar/types-of-augmented-reality}

\bibitem{ArraysListDictionaries}\textit{Packt Hub: How to use arrays, lists, and dictionaries in Unity for 3D game development} [online]. [cit. 2023-12-17]. Dostupné z: \url{https://hub.packtpub.com/arrays-lists-dictionaries-unity-3d-game-development}

\bibitem{gamePreviewOnMobile} \textit{Quickly preview your game on Android device | Unity tutorial} [online]. YouTube, 25.6.2021 [cit. 2023-12-18]. Dostupné z: \url{https://www.youtube.com/watch?v=iCXwaehzRFQ}

\bibitem{QuickOutline}\textit{Quick Outline: Particles/Effects} [online]. [cit. 2023-12-17]. Dostupné z: \url{https://assetstore.unity.com/packages/tools/particles-effects/quick-outline-115488}

\bibitem{UIMouseOver} \textit{Quick Tip: Test Mouse over UI | Unity Tutorial} [online]. YouTube, 24.5.2018 [cit. 2023-12-18]. Dostupné z: \url{https://www.youtube.com/watch?v=ptmum1FXiLE}

\bibitem{UnityRichText}\textit{Rich Text: Unity UI: 1.0.0} [online]. [cit. 2023-12-17]. Dostupné z: \url{https://docs.unity3d.com/Packages/com.unity.ugui@1.0/manual/StyledText.html}

\bibitem{ARtypes}\textit{Shopify: 5 Types of AR and How They Improve Online Shopping} [online]. [cit. 2023-12-17]. Dostupné z: \url{https://www.shopify.com/blog/types-of-ar}	

\bibitem{menu} \textit{START MENU in Unity} [online]. YouTube, 29.11.2017 [cit. 2023-12-18]. Dostupné z: \url{https://www.youtube.com/watch?v=zc8ac_qUXQY}

\bibitem{UnityAssetStore}\textit{Unity Asset Store: The Best Assets for Game Making} [Online]. [cit. 2023-12-17]. Dostupné z: \url{https://assetstore.unity.com}

\bibitem{imagetracking}\textit{Unity: Ar tracked image manager: AR Foundation: 4.0.12} [online]. [cit. 2-17]. Dostupné z: \url{https://docs.unity3d.com/Packages/com.unity.xr.arfoundation@4.0/manual/tracked-image-manager.html}

\bibitem{UnityExecution}\textit{Unity: Order of execution for event functions} [online]. [cit. 2023-12-17]. Dostupné z: \url{https://docs.unity3d.com/Manual/ExecutionOrder.html}

\bibitem{UnityReference}\textit{Unity: Welcome to the Unity Scripting Reference!} [online]. [cit. 2023-12-17]. Dostupné z: \url{https://docs.unity3d.com/ScriptReference/}

\bibitem{InstantiatingGameObjects} \textit{Unity3d with AR Foundation - How To Instantiate A Game Object Per Tracked Image?} [online]. YouTube, 24.9.2019 [cit. 2023-12-18]. Dostupné z: \url{https://www.youtube.com/watch?v=iM0ghkvsRos}

\bibitem{uvoddoAR}\textit{Úvod do tématu: Rozšířená realita (AR) ve vzdělávání: O2 Chytrá škola} [online]. [cit. 2023-12-17]. Dostupné z: \url{https://vyuka.o2chytraskola.cz/clanek/54/rozsirena-realita-ar-ve-vzdelavani}

\bibitem{XRinteractionToolkit}\textit{XR Interaction Toolkit: XR Interaction Toolkit: 0.9.4-preview} [online]. [cit. 2023-12-17]. Dostupné z: \url{https://docs.unity3d.com/tages/com.unity.xr.interaction.toolkit@0.9/manual/index.html}



\end{thebibliography}

%% obrázky 
\listoffigures

%% tabulky
\listoftables

\appendix %% začínají přílohy

\titleformat{\chapter}[block]{\scshape\bfseries\LARGE}{Příloha \thechapter}{10pt}{\vspace{0pt}}[\vspace{-22pt}] %% nastavení nadpisu u příloh


\chapter*{příloha}
\begin{figure}[h!] \centering
	\begin{subfigure}[h]{0.63\textwidth}
		\includegraphics[width=\textwidth]{image/pc.png} 
	\end{subfigure}
	\caption[3D objekt počítače]{3D objekt počítače s komponenty zahrnující: RAM, základní desku, procesor, grafickou kartu a napájecí zdroj počítače.}
	\label{fig:pocitac}
	\begin{subfigure}[h]{0.63\textwidth}
		\includegraphics[width=\textwidth]{image/checkers.png}
	\end{subfigure}
	\caption[3D objekt dámy]{3D objekt dámy.}
	\label{fig:dama}
\end{figure}

\begin{figure}[h!] \centering
	\begin{subfigure}[h]{0.63\textwidth}
		\includegraphics[width=\textwidth]{image/house.png} 
	\end{subfigure}
	\caption[3D objekt domu]{3D objekt domu z animovaného seriálu.}
	\label{fig:pocitac}
	\begin{subfigure}[h]{0.63\textwidth}
		\includegraphics[width=\textwidth]{image/tlacitka.png}
	\end{subfigure}
	\caption[3D objekt tlačítek]{3D objekt tlačítek.}
	\label{fig:tlacitka}
	\begin{subfigure}[h]{0.5\textwidth}
		\includegraphics[width=\textwidth]{image/skrtic.png} 
	\end{subfigure}
	\caption[3D objekt pneumatického škrtiče]{3D objekt pneumatické škrtiče.}
	\label{fig:skrtic}
\end{figure}



\begin{figure}[h!] \centering
	\begin{subfigure}[h]{0.63\textwidth}
		\includegraphics[width=\textwidth]{image/pist.png}
	\end{subfigure}
	\caption[3D objekt pístu]{3D objekt pístu.}
	\label{fig:pist}
	\begin{subfigure}[h]{0.63\textwidth}
		\includegraphics[width=\textwidth]{image/kladka.png} 
	\end{subfigure}
	\caption[3D objekt kladky]{3D objekt kladky.}
	\label{fig:kladka}

	
\end{figure}

\end{document}